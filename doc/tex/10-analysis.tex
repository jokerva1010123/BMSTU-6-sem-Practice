\section{Аналитическая часть}

\subsection{Формализация проблемы}
Большинство систем светофоров работают по стандартной схеме, установленной на основе исследований движения в определенное время дня или недели. Однако они не способы реагировать на ищменения в потока транспорта в реальном времени. 

Многие современные системы управления транспортными не обладают необходимой степенью автоматизации. Даже если есть счетчики движения, их данные редко используются для динамического управления светофорами.

\subsection{Подход к решению}
Как описано выше, данная работа моделирует "умную" систему светофоров на Т-образном перекрестке с пешеходными переходами. Последовательность преобразования сигнала типичного светофора можно упрощенно представить следующим образом:
\begin{itemize}
    \item водители едут;
    \item оба (водители и пешеходы) ждут;
    \item пешеходы идут;
    \item оба (водители и пешеходы) ждут;
    \item повторить цикл.
\end{itemize}

«Умный» светофор работает ни в коей мере не так. Это владение информацией об автомобилях на дороге и пешеходах на пешеходном переходе. При наличии пешеходов и автомобилей светофор работает как обычный светофор. Если машин нет, а на пешеходном переходе ждут пешеходы, то им будет горить зеленый свет до тех пор, пока не появится хотя бы одна машина, и наоборот.

Если нет ни машин, ни пешеходов, то для машин будет гореть зеленый свет, потому что им нужно больше времени, чтобы замедлиться и набрать скорость, чтобы продолжить движение.

Ниже, на таблице \ref{tab:1}, приведена функциональная логика для «умного» светофора.

\newpage

\begin{table}[H]
		\captionsetup{justification=raggedright, singlelinecheck=false}
	\caption[]{\label{tab:1} Функциональная логика для «умного» светофора}
	\begin{center}
		\begin{tabular}{|p{5cm}|p{5cm}|p{5cm}|}
			\hline
			 & Нет пешеходов & Пешеходы существуют \\
			\hline
			Нет машин & Зеленый свет для машин & Зеленый свет для пешеходов   \\
			\hline
			Машины существуют & Зеленый свет для машин  & Обычный режим светофора  \\
			\hline
		\end{tabular}

	\end{center}
\end{table}

\subsection{Потециальные сложности}
При внедрении системы интеллектуального светофора возникают следующие проблемы, которые необходимо учитывать и решать, чтобы обеспечить эффективность и безопасность системы.
\begin{enumerate}
    \item Проектирование системы. Чтобы обеспечить гибкость и масштабируемость системы, проект должен быть специально спланирован. Необходимо определить расположение светофоров с учетом транспортного потока, скорости и размера дороги, а также других факторов, таких как перекрестки, кольцевые развязки и близость к школам или жилым районам.
    \item Связь и коммуникация. "Умные" системы светофоров должны иметь возможность связываться и взаимодействовать с различными устройствами, такими как датчики, камеры наблюдения, системы управления сигналами и системы управления дорожным движением. Это требует использования общих протоколов и стандартов для обеспечения совместимости и интеграции систем.
    \item Управление и обслуживание. Интеллектуальные системы светофоров требуют регулярного управления и обслуживания для обеспечения непрерывной и эффективной работы.
\end{enumerate}
